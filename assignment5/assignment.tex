\documentclass{article}

\usepackage{graphicx}
\usepackage{amsmath}    
        
\title{Design \& Professional Skills --- Pacman Protocol Specification Assignment}      
\author{}
\date{}
        
\begin{document}

\maketitle

\section{Introduction}

In this assignment you do not need to write any code; instead you will
write a technical specification.  You have been given the source code
for an alpha-release of a multi-player networked Pacman game.  The
original Pacman game predates networked games, but we have extended
the game to allow you to compete with other players for food and to
try and tempt the ghosts to attack them.

Our version of this game was originally intended to be a sociable if
somewhat silly game to be played by two players in the same room. If
your Pacman goes down the tunnel, it will disappear from your screen
and reappear on your opponent's screen, but is still controlled by
you.  You have to look at their screen to play.

During the pandemic, not everyone could be in the same room as
one another, so a remote mode was been added (this is enabled using
the ``-r'' command line flag).  In remote mode, the game also displays a
small version of their Pacman screen, making it possible to play
against remote opponents. This does, however, reduce the sillyness somewhat.

The current version of the network protocol used to communicate between the
two computers was implemented by a professor who shall remain
anonymous, late one night after a long evening in the pub.  It works,
but it is an ugly quick and dirty solution, which does not perform
especially well, and which has potential security vulnerabilities.

Your challenge is to design a better protocol and to write a
specification of that protocol that is complete and unambiguous enough
for another programmer to be able to implement an interoperable
implementation in Python.  You do not need to write an implementation of your
protocol, though to understand the current protocol you may need to
add some debugging print statements to the existing code.

\section{Running Pacman}

In the ENGF0002 github repository, in
assignments/assignment5/multi-player/src there is source code for a
multi-player version of Pacman.  If you want the full pacman
experience with sound, use pip to install the {\tt simpleaudio} python
package.

There are several ways to run the code.

\subsection*{Client-Server Mode}

In the simplest method, one
player's computer acts as a ``server'' and the other as a ``client''.
The server must run first, and then the client connects to the server.
Once the network connection has been established, there is no
difference between how the server and client behave.

To run the server:

\begin{verbatim}
python3 pacman.py -r -s -p <passwd>
\end{verbatim}

The server pacman will start, will display its IP address on screen,
and will wait for the client to connect.  Substitute a password of your choice for {\tt <passwd>}.

To run the client:

\begin{verbatim}
python3 pacman.py -r -c <ip_address> -p <passwd>
\end{verbatim}

The IP address must be that of the server pacman, and {\tt <passwd>} must be
the same one they chose.  The password is there to give some minimal
control over who can connect to a server pacman.

You can then play multi-player pacman.  If your pacman goes through
the tunnel, it appears on the screen of your opponent, and can eat the
food, eat the powerpills, eat frightened ghosts, and die.  At present,
the two pacmen do not directly interact.  They simply pass straight
through each other.

Client-server mode is useful when both players are on the same local
network, or when the server is on a computer that is reachable from
the public Internet, rather than behind a firewall.  Unfortunately in
2020, it is rare that two players will be in the same room, but you
can run both client and server on the same computer and play against
yourself, which isn't very entertaining, but is useful for testing.

\section{Two clients, one relay server}

In {\tt assignments/assignment5/pacman-server/} there is a simple
relay server that passes messages between two clients.  You can run
the relay server as:

\begin{verbatim}
python3 pacman_server.py
\end{verbatim}

Once the server is running, two clients can both connect to the server:

\begin{verbatim}
python3 pacman.py -r -c <ip_address> -p <passwd>
\end{verbatim}

Again, both clients much choose the same password to be connected to
each other.  Many pairs of clients can all use the same server - only
pairs of clients using the same password will be connected to each
other.  The IP address must be the address of the computer running the
server.

You can run your own server on any computer that is reachable from
your clients.  Alternatively, we will run a server on a computer at
UCL (IP address will be posted on Piazza), but the latency connecting
from remote parts of the world may affect the play.

\section{About the code}

The code is a form of model-view-controller, roughly similar to that
used for Frogger.

Your pacman's interactions with its environment are always modelled on
your computer, even when it is visiting the remote screen.  This
allows fast interaction between keypresses and motion in the model,
which is necessary to turn corners precisely.  The display of your
pacman on the remote screen may lag slightly if the network is not
performing well.  Your pacman is always shown in yellow, whereas your
opponents is always pink.

Ghosts's motion and strategy are always modelled on their home
computer. Ghosts cannot traverse the tunnels and visit the remote
screen.

We use the following terminology to distinguish between visiting pacman and various game objects:
\begin{itemize}
\item{LOCAL}: the game object is a local game object, and is currently on the local screen.
\item{AWAY}: our pacman is current away on the remote screen.
\item{REMOTE}: a game object on the remote screen that our AWAY pacman might interact with.
\item{FOREIGN}: the other player's pacman, when it is visiting our screen.
\end{itemize}
In this document, when these terms are capitalized, they have these
specific meanings.  

When our pacman is AWAY, the local model needs to know about
everything it can interact with.  At game start or restart, each
computer sends the other a copy of its maze.  The game ships with
three different mazes, though more can be added.  As the maze is sent
over the network to the other player, it is fine if the mazes stored
on each computer are different.  A player can choose a maze by
selecting "-m $<$mazenum$>$" on the command line, where mazenum is an
integer, typically from 0 to 2.  If no maze is specified, it is
selected randomly.  The maze that is sent includes the location of all
the food and powerpills.

The model running on your computer keeps two mazes in memory - the
 LOCAL one and the REMOTE one.

To keep the copies of the maze running on your computer and the remote
computer synchronized, each computer continuously informs the other of
actions.  The actions include the pacman moving (position, direction
and speed), the LOCAL ghosts moving, ghosts changing state, eating
food or powerpills, ghosts getting eaten, and the player dying.

On receipt of these messages from the remote computer, the local
computer updates the relevant maze - this might be the local maze if
the other pacman is currently FOREIGN (ie it is visiting our maze), or
it might be the remote maze if the other pacman is REMOTE.

When our pacman visits the remote screen, it becomes AWAY.  Our
computer first sends a ``pacman arrived'' message, so the remote
computer can initialise any state.  Whenever our pacman moves, our
computer sends ``pacman update'' messages to the remote computer,
giving the current position, direction and speed of our pacman.  It
does this irrespective of whether the pacman is LOCAL or AWAY.

Whenever our LOCAL or AWAY pacman eats food or powerpills, this is
detected by the model running on our own computer, using its copy of
the LOCAL or REMOTE maze as appropriate.  Our computer then sends an
``eat'' update message to the remote computer informing it that food
or a powerpill has been eaten.  Thus, even when our pacman is AWAY,
interactions between it and food are still handled by the local model.

Our computer also sends ``ghost update'' messages whenever the LOCAL
 ghosts move.  These give the position, direction, speed, and mode of
 each ghost.  Amongst other things, mode includes whether the ghost is
 in ``FRIGHTEN'' mode (having turned blue, and being edible).  The
 remote computer also sends such "ghost update" messages to our
 computer. Our model uses this information to update a local model of
 the REMOTE ghosts to determine if our pacman was either killed by
 one, or has eaten one.

If our model detects that our AWAY pacman has eaten a REMOTE ghost
 (while it was in FRIGHTEN mode), it sends a ``foreign pacman ate
 ghost'' message to update the remote system.

If our model detects that our AWAY pacman was killed by a REMOTE
ghost, it sends a ``foreign pacman died'' message.

If our model detects that our AWAY pacman has traversed the tunnel
 again, and is now LOCAL, it sends a ``foreign pacman left'' message.
  The remote side will stop displaying the pacman on its main screen
 (and display it again on the smaller remote screen if enabled).

Some events require that our AWAY pacman be forcibly sent home.  This
happens when the level is completed on the remote screen, for example.
The remote system sends a ``pacman go home'' message.  Our system then
resets our pacman to LOCAL, and sends a ``foreign pacman left''
message in reply.

Whenever our pacman's score changes, whether our pacman is LOCAL
or AWAY, our system sends the remote system a ``score update''
message.

The local game board also has states associated with it.  These are defined the class GameMode in {\tt pa\_model.py}:

\begin{itemize}
\item {\tt STARTUP}
\item {\tt CHASE}
\item {\tt FRIGHTEN}
\item {\tt GAME\_OVER}
\item {\tt NEXT\_LEVEL\_WAIT}
\item {\tt READY\_TO\_RESTART}
\end{itemize}

Changes between these states are communicated using ``status update'' messages.

Gameplay only happens in {\tt CHASE} and {\tt FRIGHTEN} state (the difference being
whether a powerpill has recently been eaten).  The software is in
{\tt STARTUP} state while playing the startup jingle.

If either player loses their last life, the game ends.  The losing
player's computer goes to {\tt GAME\_OVER} state, and sends a status update
message.  The other side then also moves to {\tt GAME\_OVER} state.

From {\tt GAME\_OVER} state, if the local player presses ``r'' to restart,
the local computer goes to {\tt READY\_TO\_RESTART} state and sends an update
message.  The game restarts when the second player also presses ``r'',
and sends a replying ``{\tt READY\_TO\_RESTART}'' status update.

When a  level is  cleared on  a screen, that  screen's system  goes to
{\tt NEXT\_LEVEL\_WAIT} while it  plays the jingle and the  player gets ready.
Completing a level does not affect the level being played on the other
screen, except the pacmen positions are reset.

The complete list of messages in the current version of the protocol is therefore:
\begin{enumerate}
\item {\bf maze update}
\item {\bf pacman arrived}
\item {\bf pacman left}
\item {\bf pacman died}
\item {\bf pacman go home}
\item {\bf pacman update}
\item {\bf ghost update}
\item {\bf ghost was eaten}
\item {\bf foreign pacman ate ghost}
\item {\bf eat}
\item {\bf score update}
\item {\bf lives update}
\item {\bf status update}
\end{enumerate}

\subsection{Existing networking code}

The existing networking code is in {\tt pa\_network.py}.  The protocol
follows the above description, but is not a good implementation.  It
uses a TCP\footnote{we will discuss TCP and UDP in course materials} connection for communication, uses verbose message names,
and uses {\tt pickle} to encode and decode payloads.  You should probably
pay attention to the code in {\tt check\_for\_messages()}, which ensures that
when multiple messages are received by one call to {\tt recv()}, the
additional messages are not missed, but are kept and processed.

This protocol is less than ideal for a number of reasons:

\begin{itemize}
\item The encoding is python specific.  Your brief is to produce a specification that could be implemented using any programming language.
\item Pickle is not robust to malicious input.  Your brief is to
  produce a specification that indicates how received values should be
  sanity-checked.  For example, a ghost number of 5 would be illegal.
\item The protocol is a strange mixture of binary encoding for message length, text encoding for message type, and binary pickle encoding.  This is really ugly!  Your brief is to write a specification that is clean.  You can produce either a text-based protocol, or a binary-encoded one, but be consistent.
\item The protocol is chatty.  For example, it sends multiple messages
  for each video frame, including, for example, four separate ghost
  update messages.  If you wish to combine multiple updates into one
  message, you may do so.
\item The protocol only uses TCP.  For some messages, such as sending
  the maze, this is sensible.  For others, UDP would provide more
  timely delivery.  Your protocol can use TCP, UDP, or both.  Bear in
  mind though that if you use both, you may have to worry about
  message ordering between UDP and TCP connections (UDP messages may
  overtake TCP ones for example).  If you choose UDP only, you must
  state how missing packets will be handled.
\end{itemize}

\section{Your Task}

\begin{itemize}
\item Your task is to write a protocol specification for a protocol to
replace the one above.  You must not directly use any existing
protocol, but you can modify such a protocol if you wish.  However,
your protocol specification cannot simply reference an external
protocol specification - it should be possible to implement your
protocol without reading any other specification other than TCP or UDP.

\item Your protocol may use TCP, UDP, or both.

\item Your protocol should use either a text-based encoding or a binary
encoding.  You should probably not mix the two without good reason (if you have
a good reason, you should explain it).

\item Your protocol must not use pickle.

\item Your protocol must not use HTML, XML, JSON, or any other
  pre-packaged format that means you don't need to think about the
  problem for yourself.

\item You need to specify how to encode the information currently sent
 in all 12 existing message types, though your protocol does not
 necessarily need to have 12 distinct messages.

\item You need to specify any additional processing the receiver
 should perform on receipt of these messages {\em that is not already
 performed in the existing code}.  You do not need to specify
 processing that the existing code already performs, such as what the
 Model does with a particular message type.  Examples of additional
 processing might be retransmitting lost data when using UDP.

\item You may use text from this document if it helps.

\item Hint: if you don't understand how a particular message type
  should work, or what the values included are, try adding print
  statements to {\tt pa\_network.py} to show the message contents
  before encoding or after decoding.

\item You may ask on Piazza for more information about how things
  work, but I will only answer public questions, so everyone has the
  same information.  It is a normal part of writing a specification to
  gather information, and I will be happy to clarify anything that is
  unclear about how the existing protocol works, or how the game
  works.

\item Remember: {\em The documentation is always wrong!} If there are
  differences between how the python code of the existing protocol
  works and how it is described in this document, the python code is
  authoritative and takes priority (there are no deliberate
  differences, but errors may have crept in).
\end{itemize}

\newpage
\section{Marking}

You will be arranged in groups, and will assign a mark to each other
person's specification from your group.  When marking, you should
consider the difficulty you would have understanding the specification
well enough to implement it in python.

When you are marking, you are not marking the quality or grammatical correctness of the English,
so long as it is intelligible and unambiguous.

You should assign marks for:
\begin{itemize}
  \item Conciseness.  Don't waffle.  Be specific.  This isn't a work of literature.
  \item Correctness.  Will the protocol fail if implemented as specified.
  \item  Unambiguous.  Do you understand how to code what is specified in all cases?
  \item  Completeness. Are some things missing?
  \item  Examples. Although examples are not part of the specification (in technical terms, they're ``non-normative''), use of a few examples may be useful in complex cases.  But not to the extent this severely contradicts conciseness.
\end{itemize}

The expectation is that for most groups, a median mark for this
coursework will be between 60 and 70\%, representing a 2:1 grade.
More complete marking guidance will follow.

\end{document}
  

